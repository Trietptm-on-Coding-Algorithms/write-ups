%!TEX root = f4b_xor_w4kfu.tex
%!TEX program = lualatex
%!TEX encoding = utf-8
%!TEX spellcheck = en_US

\documentclass{easychair}
% !TeX root = design.tex
%!TEX program = lualatex


\usepackage[utf8]{inputenc}
\usepackage[T1]{fontenc}

\usepackage{fontspec}
\usepackage{lmodern}

% add page number
% \pagestyle{headings}

% beamer setting
% \usetheme{CambridgeUS}
% \setbeamertemplate{footline}[page number]
% \setbeamertemplate{blocks}[rounded][shadow=true]

% \pdfoptionpdfminorversion=7
% \pdfminorversion=6

% use table of content for each chapter
% \usepackage{minitoc}

% \usepackage[a-3u]{pdfx}

% \usepackage[toc,page]{appendix}
\usepackage[titletoc]{appendix}

% csquotes
\usepackage{csquotes}

% use biblatex
\usepackage[sorting=anyt,sortcites=true,backend=biber,url=false,doi=false,maxnames=7,isbn=false,firstinits=true,style=numeric]{biblatex}
% \addbibresource{reference}
\bibliography{reference}


% encoding (just for pdflatex)
% \usepackage[latin1]{inputenc}

% hyphenating
\usepackage[english]{babel}
% \usepackage[autostyle]{csquotes}

% AMS Math
\usepackage{braket}
% \usepackage{amsmath,amssymb,amsthm,mathtools}
% \usepackage{breqn}
\usepackage{amsmath,amssymb,mathtools,amsthm}
% \usepackage{unicode-math}
% \usepackage{lualatex-math}
% \usepackage{stix}
% \usepackage{cmbright}
% \usepackage{lmodern}
% \usepackage{lxfonts}
\usepackage{nicefrac}
\usepackage{stmaryrd}
\usepackage{xfrac}
% \usepackage{MnSymbol}


% for algorithm
%\usepackage{algorithm}
%\usepackage{algpseudocode}
\usepackage{algorithm2e}
\SetAlFnt{\small}

% hyperref
\usepackage{hyperref}
% \usepackage[pdfa]{tulhypref}
\hypersetup
{
%   bookmarks       = true,
  unicode         = true,
  pdftitle        = Maze challenge,
  pdfauthor       = Verimag, %auteur du document
  pdfsubject      = Maze challenge,
  pdftoolbar      = true, %barre d'outils non visible
  pdfmenubar      = true, %barre de menu visible
  pdfhighlight    = /O, %effet d'un clic sur un lien hypertexte
  colorlinks      = true, %couleurs sur les liens hypertextes
%   pdfpagemode = None, %aucun mode de page
%   pdfpagelayout = SinglePage, %ouverture en simple page
  pdffitwindow    = true, %pages ouvertes entierement dans toute la fenetre
  linkcolor       = violet, %couleur des liens hypertextes internes
  citecolor       = blue, %couleur des liens pour les citations
  urlcolor        = orange, %couleur des liens pour les url
%   pagebackref=true
%   bookmarksdepth  = 3
}

% \usepackage[sc]{mathpazo}
% \newfontfamily\DejaSans{DejaVu Sans}

\usepackage[usenames,dvipsnames,svgnames,table]{xcolor}
\definecolor{darkgoldenrod}{rgb}{0.72, 0.53, 0.04}
\definecolor{darkorchid}{rgb}{0.6, 0.2, 0.8}
% for listings
\usepackage{listings}
\lstset{basicstyle=\footnotesize\ttfamily,showspaces=false,frame=tb}
% \ltset{basicstyle=\small\ttfamily,showspaces=false,frame=tb}
\lstloadlanguages{C++,[x86masm]Assembler}

% graphicx and caption
\usepackage{graphicx}
% \usepackage{bmpsize}
% \usepackage[font={small}]{caption}
\usepackage{caption}
% \usepackage{subfig}
% \usepackage[config,position=t]{subfig}
\usepackage{subcaption}
% \captionsetup{compatibility=false}
% \DeclareCaptionLabelSeparator{periodspace}{.\quad}
% \captionsetup{font=footnotesize,labelsep=periodspace,singlelinecheck=false}
% \captionsetup[sub]{font=footnotesize,singlelinecheck=true}
% \renewcommand\thesubfigure{(\alph{subfigure})}
% \usepackage{rotating}
\usepackage{pdfpages}

% for smarter reference
\usepackage{cleveref}

% reference depth
\setcounter{secnumdepth}{2}

% for appendix
% \usepackage{appendix}

% for footnote
\usepackage{fancyvrb}
\VerbatimFootnotes

% for diagram
\usepackage{tikz}
\usetikzlibrary{calc,
                automata,
                arrows,
                matrix, 
                plotmarks,
                positioning,
                shapes,
                shapes.multipart,
                shapes.misc,
                decorations, 
                decorations.markings, 
                decorations.pathreplacing, 
                decorations.pathmorphing,
                backgrounds,
                fit,
                patterns}
% \usepackage{tkz-base}

% for commutative diagram
\usepackage{tikz-cd}
%\tikzset{commutative diagram/row}

\usepackage{wrapfig}

% for margin note
\usepackage{marginnote}

% for itemize
\usepackage{enumerate}
\usepackage[inline]{enumitem}
% \setitemize{label=\usebeamerfont*{itemize item}%
%   \usebeamercolor[fg]{itemize item}
%   \usebeamertemplate{itemize item}}

% create procedure environment
%\newtheorem{procedure}{Procedure}

% some tikz styles
% \tikzset{morphism/.style={circle,draw,thick,align=center,inner sep=0pt,minimum size=12pt}}
% \tikzset{bigmorphism/.style={circle,draw,thick,align=center,inner sep=0pt,minimum size=25pt}}
%\tikzset{objects/.style=}

% enumitem
% \usepackage{enumitem}

\usepackage{fancyvrb}

\usepackage{cprotect}

\usepackage{footmisc}

\usepackage{array}

% \usepackage{gnuplottex}

\usepackage{float}

\usepackage{mathrsfs}

\usepackage[export]{adjustbox}

\usepackage{rotating}

% for footnote
\usepackage{perpage} 
\MakePerPage{footnote}

\usepackage{parcolumns}

\usepackage{url}

% \usepackage{savetrees}

% \usepackage{floatrow}
% \floatsetup[table]{font=small}

% \newtheorem{principle}{Principle}
% \crefname{principle}{principle}{principles}
% \Crefname{principle}{Principle}{Principles}

\theoremstyle{plain}

% \newtheorem{hypothesis}{Hypothesis}
% \crefname{hypothesis}{hypothesis}{hypotheses}
% \Crefname{hypothesis}{Hypothesis}{Hypotheses}
% 
% \newtheorem{assumption}{Assumption}
% \crefname{assumption}{assumption}{assumptions}
% \Crefname{assumption}{Assumption}{Assumptions}
% 
% \newtheorem{heuristic}{Heuristic}
% \crefname{heuristic}{heuristic}{heuristics}
% \Crefname{heuristic}{Heuristic}{Heuristics}


\newtheorem{theorem}{Theorem}
\crefname{theorem}{theorem}{theorem}
\Crefname{theorem}{Theorem}{Theorems}

% \newtheorem{proposition}{Proposition}[section]
\newtheorem{proposition}{Proposition}
\crefname{proposition}{proposition}{propositions}
\Crefname{proposition}{Proposition}{Propositions}
% 
\newtheorem{corollary}{Corollary}[section]
\crefname{corollary}{corollary}{corollaries}
\Crefname{corollary}{Corollary}{Corollaries}

\newtheorem{problem}{Problem}
\crefname{problem}{problem}{problems}
\Crefname{problem}{Problem}{Problems}

\newtheorem{lemma}{Lemma}[section]
\crefname{lemma}{lemma}{lemmas}
\Crefname{lemma}{Lemma}{Lemmas}

\newtheorem{observation}{Observation}[section]
\crefname{observation}{observation}{observations}
\Crefname{observation}{Observation}{Observations}

\theoremstyle{definition}
% 
\newtheorem{definition}{Definition}[section]
\crefname{definition}{definition}{definitions}
\Crefname{definition}{Definition}{Definitions}
% 
\newtheorem{hypothesis}{Hypothesis}
\crefname{hypothesis}{hypothesis}{hypotheses}
\Crefname{hypothesis}{Hypothesis}{Hypotheses}

% \newtheoremstyle{dotlessP}{}{}{}{}{\color{blue}\bfseries}{}{ }{}

% \theoremstyle{dotlessP}
% \theoremstyle{plain}
\theoremstyle{remark}

% \newtheorem{example}{Example}[section]
\newtheorem{example}{Example}
\crefname{example}{example}{examples}
\Crefname{example}{Example}{Examples}

\theoremstyle{remark}

% \newtheorem{remark}{Remark}[section]
\newtheorem{remark}{Remark}
\crefname{remark}{remark}{remarks}
\Crefname{remark}{Remark}{Remarks}

\newtheorem{note}{Note}[section]
\crefname{note}{note}{notes}
\Crefname{note}{Note}{Notes}

\crefname{secinapp}{appendix}{appendices}
\Crefname{secinapp}{Appendix}{Appendices}

% \crefname{appendix}{appendix}{appendices}
% \Crefname{appendix}{Appendix}{Appendices}
% \newtheorem{corollary}{Corollary}

% \newtheoremstyle{dotlessP}{}{}{}{}{\color{blue}\bfseries}{}{ }{}
% \theoremstyle{dotlessP}

%%%% BINARY INSTRUCTION %%%%
% \newcommand{\jump}{{\tt jump}\xspace}
% \newcommand{\call}{{\tt call}\xspace}
% \newcommand{\ret}{{\tt ret}\xspace}
% \newcommand{\jcc}{{\tt jcc}\xspace}
% \newcommand{\instr}{{\tt i}\xspace}
% \newcommand{\code}{{\tt c}\xspace}
% 
% %%%% GENERAL %%%%%%%%%%
% \newcommand{\dcfg}[1]{{\cal G}_{#1}}
% \newcommand{\E}{{\cal E}}
% \newcommand{\V}{{\cal V}}
% % \newcommand{\T}{{\cal T}}
% \newcommand{\N}{{\mathbb{N}}}

% \newtheorem{problem}{Problem}
% \crefname{problem}{problem}{problems}
% \Crefname(problem}{Problem}{Problems}

% \makeatletter
% \let\@twosidetrue\@twosidefalse
% \let\@mparswitchtrue\@mparswitchfalse
% \makeatother

% \usepackage[left=0.5cm,right=0.5cm,top=0.5cm,bottom=0.5cm]{geometry}

% \vspace{-2.5mm}
% \setlength{\textfloatsep}{\baselineskip plus 0.2\baselineskip minus 0.2\baselineskip}
% \setlength{\textfloatsep}{10pt}
% \setlength{\belowcaptionskip}{-10pt}
% \setlength{\intextsep}{10pt}
% \setlength\abovedisplayskip{0pt}
% \usepackage[font=small,skip=10pt]{caption}
\allowdisplaybreaks

% \usepackage{titlesec}
% \titlespacing\section{0pt}{12pt plus 4pt minus 2pt}{0pt plus 2pt minus 2pt}

% \setlength{\parindent}{0pt}
% \usepackage[nodisplayskipstretch]{setspace}
% \setstretch{1.5}

% \definecolor{cornflowerblue}{rgb}{0.39, 0.58, 0.93}

% \usepackage{lmodern}
% \usepackage[sc]{mathpazo}
% \linespread{1.05}
% \usepackage{baskervald}
% \usepackage{nbaskerv}
% \usepackage[utf8]{inputenc}
% \usepackage[T1]{fontenc}
% \usepackage{luatextra}
% \defaultfontfeatures{Ligatures=TeX}

% \usepackage{unicode-math}
% \usepackage{lualatex-math}
% \usepackage{charter}
% \usepackage[expert]{mathdesign}

\usepackage{afterpage}
\usepackage{changepage}

\usepackage{minted}
\usemintedstyle{friendly}
\usepackage{tcolorbox}
\usepackage{etoolbox}
% \BeforeBeginEnvironment{minted}{\begin{tcolorbox}}%
% \AfterEndEnvironment{minted}{\end{tcolorbox}}%

% \setcounter{totalnumber}{1}
% \setcounter{topnumber}{1}
% \setcounter{bottomnumber}{1}
% \renewcommand{\topfraction}{.99}
% \renewcommand{\bottomfraction}{.99}
% \renewcommand{\textfraction}{.01}

% \makeatletter
% \newcommand*{\twopagepicture}[4]{%
%     \checkoddpage
%     \ifoddpage
%         \expandafter\@firstofone
%     \else
%         \expandafter\afterpage
%     \fi
%     {\afterpage{%
%     \if #1t%
%         \if #2p%
%             \thispagestyle{empty}%
%             \afterpage{\thispagestyle{empty}}%
%         \fi
%     \fi
%     \begin{figure}[#1]
%         \if #2p%
%             \if #1t%
%                 \vspace*{-\dimexpr1in+\voffset+\topmargin+\headheight+\headsep\relax}%
%             \fi
%         \fi
%         \if #1b%
%             \caption{#4}%
%         \fi
%         \makebox[\textwidth][l]{%
%         \if #2p\relax
%             \let\mywidth\paperwidth
%             \hskip-\dimexpr1in+\hoffset+\evensidemargin\relax
%         \else
%             \let\mywidth\linewidth
%         \fi
%         \adjustbox{trim=0 0 {.5\width} 0,clip}{\includegraphics[width=2\mywidth]{#3}}}%
%         \if #1b\else
%             \caption{#4}%
%         \fi
%         \if #2p%
%             \if #1b%
%                 \vspace*{-\dimexpr\paperheight-\textheight-1in-\voffset-\topmargin-\headheight-\headsep\relax}%
%             \fi
%         \fi
%     \end{figure}%
%     \begin{figure}[#1]
%         \if #2p%
%             \if #1t%
%                 \vspace*{-\dimexpr1in+\voffset+\topmargin+\headheight+\headsep\relax}%
%             \fi
%         \fi
%         \makebox[\textwidth][l]{%
%         \if #2p%
%             \let\mywidth\paperwidth
%             \hskip-\dimexpr1in+\hoffset+\oddsidemargin\relax
%         \else
%             \let\mywidth\linewidth
%         \fi
%         \adjustbox{trim={.5\width} 0 0 0,clip}{\includegraphics[width=2\mywidth]{#3}}}%
%         \if #2p%
%             \if #1b%
%                 \vspace*{-\dimexpr\paperheight-\textheight-1in-\voffset-\topmargin-\headheight-\headsep\relax}%
%             \fi
%         \fi
%     \end{figure}%
%     }}%
% }
% \makeatother

% \usepackage[usenames]{color}

% some tikz styles
\tikzset{morphism/.style={circle,draw,thick,align=center,inner sep=0pt,minimum size=12pt}}
\tikzset{bigmorphism/.style={circle,draw,thick,align=center,inner sep=0pt,minimum size=25pt}}

\usepackage{setspace}
% \onehalfspacing

\def \i {\texttt{i}}
\def \ti {\textnormal{\texttt{i}}}
\def \Buf {\texttt{Buf}}
\def \tB {\texttt{B}}
\def \tBuf {\textnormal{\texttt{Buf}}}
\def \Addrs {\texttt{Addrs}}
\def \Bytes {\texttt{Bytes}}
\def \tBytes {\textnormal{\texttt{Bytes}}}

\def \tH {\texttt{H}}
\def \tT {\texttt{T}}
\def \tP {\texttt{P}}
\def \tU {\texttt{U}}
\def \tE {\texttt{E}}
\def \tF {\texttt{F}}
\def \tR {\texttt{R}}
\def \tS {\texttt{S}}
\def \tA {\texttt{A}}
\def \tY {\texttt{Y}}
\def \tP {\texttt{P}}
\def \tW {\texttt{W}}
\def \tV {\texttt{V}}
\def \tC {\texttt{C}}
\def \tD {\texttt{D}}
\def \tL {\texttt{L}}
\def \tI {\texttt{I}}

\def \tLF {\texttt{\textbackslash n}}
\def \tCR {\texttt{\textbackslash r}}
\def \tSP {\texttt{SPACE}}

\def \cA {\mathcal{A}}
\def \cC {\mathcal{C}}
\def \cD {\mathcal{D}}
\def \cM {\mathcal{M}}
\def \cN {\mathcal{N}}
\def \cR {\mathcal{R}}

% styles copied from http://tex.stackexchange.com/questions/168314/how-to-use-latex-commands-to-draw-a-flowchart
% \tikzset{
%   decision/.style = {
%     diamond,
%     draw
%   }
% }

\tikzset{decision/.style = 
  {
    diamond, draw, text width=7em, text badly centered, inner sep=0pt
  }
}

\tikzset{block/.style=
  {
    rectangle, draw, text width=17em, text centered, rounded corners
  }
}

\tikzset{ablock/.style=
  {
    rectangle, draw, text width=10em, text centered, rounded corners
  }
}

\tikzset{bblock/.style=
  {
    rectangle, draw, text width=8em, text centered, rounded corners
  }
}

\tikzset{cblock/.style=
  {
    rectangle, draw, text width=20em, text centered, rounded corners
  }
}

\tikzset{sblock/.style=
  {
    rectangle, draw, fill=blue!25, text width=4em, text centered, rounded corners
  }
}

\tikzset{descr/.style = 
  {
    fill = white, inner sept = 2.5pt
  }
}

\tikzset{connector/.style = 
  {
    -latex, font = \scriptsize
  }
}

\tikzset{rectangle connector/.style = 
  {
    connector, to path={(\tikztostart) -- ++(#1,0pt) \tikztonodes |- (\tikztotarget)}, pos = 0.5
  }
}

\tikzset{rectangle connector/.default = -2cm}

\tikzset{straight connector/.style = 
  {
    connector, to path=--(\tikztotarget) \tikztonodes
  }
}

\begin{document}

\title{Deobfuscating an ``onion obfuscated'' challenge}
\titlerunning{F4b\_XOR\_F4kfu\_Writeup}

\author{
  Thanh Dinh Ta\inst{1}
}
\institute{
  Tetrane
}
\authorrunning{T. D. Ta}

\clearpage
\maketitle

\begin{abstract}
  The reverse engineering challenge \texttt{F4b\_XOR\_W4kfu} proposed in the CTF of Grehack $2015$ captures our interest since its novel code protection techniques. We present in this writeup a semi-automated approach to deobfuscate the challenge using both dynamic and static analysis.
  % This writeup presents our approach and result obtained in analyzing the challenge.
\end{abstract}

\section{Introduction}
\label{sec:introduction}

\texttt{F4b\_XOR\_W4kfu} is a 32 bits \texttt{PE} binary, it is the challenge of the highest point over all categories of Grehack $2015$ CTF\footnote{\url{https://grehack.fr/2015/ctf}}:~cryptography, exploit, reverse engineering, etc. At first sight, it seems to be a classic reverse engineering challenge; without any fancy GUI, the program simply requests for an input from the standard input, then prints \texttt{Nop!} or \texttt{Yes!} depending on whether the input is correct or not. A small notice is that it runs a little bit slow.

When analyzing the binary more carefully, we soon recognized that it is very sophisticated (this might be the reason why the challenge is worth $500$ points). First, when disassembling the binary, most of its codes is encrypted: the visible instructions only disclosure that the program read a string from the standard input. Second, when tracing its execution, the obtained trace is extremely long (that explains why the program runs slowly): we have to stop tracing when the number of traced instructions is about $5.000.000$, and this number is only a very small part of the full trace.

We cannot find any ``easy point'' which helps to get some information about the correct password. The simple \emph{side-channel attack} based on execution traces does not work since the number of executed instructions is identical for each password verification session. The \emph{differential attack} which compares different execution traces is not scalable since the traces are so long: by \emph{dynamic tainting analysis}, we can observe in the partial execution trace some positions which access/modify the input password buffer, but we cannot figure out the password verifying algorithm since these positions are not closed together, they are instead spread along the trace, moreover the number of instructions between two consecutive positions is huge. Some effort using \emph{concolic execution} tools with \emph{SMT solvers} gives no result, a trace of millions instruction goes far beyond the capability of any state of the art automated tool.

There is no space for ``low hanging fruit'' attack, being curious about how the binary works, we decided to deofuscate it and the result makes us surprise. The binary implements very interesting and novel obfuscation techniques. Its original codes are protected by \emph{nested virtual machines} consisting of two different layers. The first layer is a virtual machine where \emph{every instruction is control-flow instruction}, each is sheltered by the mechanism of decrypt-execute-encrypt \emph{self-modifying code}: the instruction is decrypted just before executing and re-encrypted immediately after. The second layer simulates a \emph{preempetive multitasking operating system}, it consists of $7$ processes running simultaneously and communicating using shared memory. The password verification algorithm is a \emph{concurrent hash function} which is spread over $7$ processes, but it contains a trapdoor making the inversion become easier once the trapdoor is detected.

% \emph{nested virtual machines} with a simulated \emph{multiple processes}\footnote{Based on the information from the correct password, we think that the authors of this challenge may call this model ``multiple threads''.} execution model, and a \emph{backdoored hash function}.

To our best knowledge, analyzing these schemes of obfuscation is not studied yet. Current research on code unpacking supposes that there exists always a \emph{transition point} from that instructions are not encrypted anymore (i.e.~the program jumps to the original entry point of the protected code), whereas in this binary there is no such transition point. Techniques on virtualization deobfuscation consider only \emph{single layer} virtual machines with simple \emph{switch based} model. Moreover, ``functions'' of programs in virtual machines do not follow to the ABI of any high-level language, consequently current decompilation techniques (e.g.~value-set based structural analysis) cannot give useful information. And yet, there is no research considering the situation where these obfuscation techniques are combined. \textcolor{red}{(many citations needed for this paragraph)}

\paragraph{Contribution}
In this writeup, we present a semi-automated approach based on \emph{trace analysis} and \emph{structural analysis} to deobfuscate \texttt{F4b\_XOR\_W4kfu}. To handle the complexity of obfuscated codes,  we first consider traces of execution to get a \emph{partial form} of embedded virtual machines and recognize tricks used to transfer the control flow. This step gives patterns to simplify the trace, consequently we get a simplified but still partial form of virtual machine. Next, we map recognized patterns to a high-level representation to get the \emph{execution models} of virtual machine, a \emph{complete decompilation} for virtual machines thanks to structural analysis is the result of this step. Once the processes are fully decompiled, the hash algorithm is \emph{sequentialized} thanks to its interleaving semantics. Finally, we build the trace formula based on the instructions of the low-layer virtual machine, this trace is much more simplified than the original traces of the program, a \emph{trapdoor function} helps to simplify considerably the trace, then a SMT solver can be used to calculate the good input password.

% existing binary code analysis/debugging tools (e.g.~IDA, WinGdb) are helpful but considered as referential tools only.
% we instead introduce REVEN, a \emph{heavyweight dynamic analysis tool} for binary codes and several plugins of it.
\end{document}